\documentstyle[12pt]{article}

\oddsidemargin 0pt
\evensidemargin 0pt
\textwidth 16.2cm

\newcommand{\vect}[1]{{\bf #1}}
\newcommand{\tens}[1]{{\bf #1}}
\newcommand{\gvect}[1]{\mbox{\boldmath $#1$}}
\newcommand{\gtens}[1]{\mbox{\boldmath $#1$}}
\newcommand{\ti}[2]{\tens{#1}^{#2}}
\newcommand{\gti}[2]{\gtens{#1}^{#2}}
\newcommand{\vr}{\vect{r}}
\newcommand{\vR}{\vect{R}}
\newcommand{\vE}{\vect{E}}
\newcommand{\vv}{\vect{v}}
\newcommand{\vF}{\vect{F}}
\newcommand{\vU}{\vect{U}}
\newcommand{\tzeta}{\gtens{\zeta}}
\newcommand{\tmu}{\gtens{\mu}}
\newcommand{\prog}[1]{{\tt #1}}
\newcommand{\name}[1]{{\sc #1}}
\newcommand{\btitle}[1]{{#1,}}
\newcommand{\vol}[1]{{\bf #1}}

\newcommand{\wsp}{\hspace*{0.5cm}}

\begin{document}

\title{HYDROLIB: installation and usage guide}
\author{K. Hinsen\\
        Institut f\"ur Theoretische Physik A\\
        RWTH Aachen}
\date{}
\maketitle
\tableofcontents
\newpage

\section{Overview}

The Hydrodynamic Interactions Library (\prog{HYDROLIB}) consists of a
set of Fortran subroutines that calculate hydrodynamic interactions in
an arbitrary configuration of equal hard spheres with stick boundary
conditions, optionally with arbitrary rigid links between the
spheres. The quantities that can be calculated are:
\begin{itemize}
\item the friction matrix for a given particle configuration
\item the mobility matrix for a given particle configuration
\item the forces on all particles of a given configuration moving
with given velocities
\item particle velocities for a given particle configuration and a given set
of forces
\end{itemize}
The main features of the library are:
\begin{itemize}
\item controllable and adjustable accuracy
\item efficient calculation
\item suitability for finite and periodic geometries
\end{itemize}

The program has been checked intensively for various configurations,
ranging from two spheres (for which it produces exact results) to more
than a hundred. In addition, many internal consistency checks have
been done and the code has been checked with
\prog{ftnchek}.\footnote{\prog{ftnchek} checks Fortran programs for
semantic errors that compilers normally don't detect. It is available
by \prog{ftp} from \prog{netlib.att.com}.} Nevertheless, I cannot
guarantee for anything; if you use \prog{HYDROLIB}, you do so at your
own risk.


\section{Friction and mobility matrices}

\subsection{Physical background}

When spheres immersed into a liquid move with prescribed linear
velocities $\vect{U}$ and angular velocities $\gvect{\Omega}$, the
forces $\vect{F}$ and torques $\vect{T}$ acting on them are described
by the {\em friction matrix} $\tzeta$:
\[
\left(\begin{array}{c} \vect{F}\\ \vect{T} \end{array}\right) = 
\left(\begin{array}{cc}
\tzeta^{tt} & \tzeta^{tr} \\
\tzeta^{rt} & \tzeta^{rr}
\end{array}\right) \left(\begin{array}{c}
\vv_0-\vU\\ \frac{1}{2}\nabla\times\vv_0-\gvect{\Omega}
\end{array}\right)
\]
The velocity field $\vv_0$ is the incident velocity field, to be
evaluated at the particle centers.

Force and torque vectors are described by three components for each
particle, therefore the friction matrix has the size $6N\times 6N$,
where $N$ is the number of particles. The friction matrix can be shown
to be symmetric.

When the forces and torques are given, the velocities can be obtained
from the
{\em mobility matrix} $\tmu$:
\[
\left(\begin{array}{c}
\vv_0-\vU\\ \frac{1}{2}\nabla\times\vv_0-\gvect{\Omega}
\end{array}\right) = 
\left(\begin{array}{cc}
\tmu^{tt} & \tmu^{tr} \\
\tmu^{rt} & \tmu^{rr}
\end{array}\right) \left(\begin{array}{c}
\vect{F}\\ \vect{T}
\end{array}\right)
\]
The mobility matrix is the inverse of the friction matrix.

A more detailed description of hydrodynamic interactions and an explanation
of the theory on which the calculations are based can be found in
\cite{cifehiwabl}. A description of this implementation has been
published in \cite{hi1}.

\subsection{Units}

Dimensionless quantities are used for all numerical calculations.
They are obtained as follows:
\begin{itemize}
\item[-] the radius $a$ of the spheres is set to 1.
\item[-] the viscosity $\eta$ of the liquid is set to $1/4\pi$.
\end{itemize}
This leaves one unit undetermined (e.g. velocity or force), which can
be chosen arbitrarily since it affects only the velocities and forces,
but not their ratios, i.e.\ the friction and mobility matrices are
independent of this choice.

\subsection{Numerical calculation}
\label{sec:numcalc}

The numerical calculation of hydrodynamic interactions is based on a
multipole expansion to take into account long-range interactions, and
on results from lubrication theory to correctly include short-range
interactions. The accuracy of the results can be controlled with the
integer parameter $l_{\rm max}$, which is related to the highest
multipole order included in the calculation. The parameter $l_{\rm
max}$ must be in the range $0\ldots 3$. Increasing this parameter
gives better accuracy, but increases the amount of memory and CPU time
needed by the program. The parameter $l_{\rm max}$ is related to
the multipole order $L$ used in \cite{cifehiwabl} and \cite{hi}
by $l_{\rm max} = L-1$.

The following table indicates how much memory is needed for a given
value of $l_{\rm max}$, expressed in terms of the number of unknowns
per particle in the system of equations that must be solved, and a
rough estimate of the accuracy of the friction matrix elements:

\begin{center}
\begin{tabular}{|l|r|l|}
\hline
$l_{\rm max}$ & Unknowns & Accuracy\\
\hline
0 & 9 & 10--20\% \\
1 & 24 & 1--5\% \\
2 & 45 & $< 1\%$ \\
3 & 84 & $<< 1\%$ \\
\hline
\end{tabular}
\end{center}

It is recommended to use $l_{\rm max} > 1$ only, since otherwise not
all long-range terms are taken into account (see \cite{cifehiwabl} for
details).


\section{Installation and usage}

\subsection{System requirements}

\prog{HYDROLIB} can be used with most standard Unix systems.
In addition to standard Unix tools, it needs a C preprocessor and a
Fortran compiler. Unfortunately, Fortran compilers vary a lot in their
abilities and command line syntax; some configuration is therefore
unavoidable.

You also need a linear algebra library. Currently only \prog{LAPACK}
is supported.\footnote{The source code contains optional calls to
\prog{IMSL}, dating back to early versions of this library.  However,
they have not been updated to the current versions of
\prog{HYDROLIB} and \prog{IMSL}, since \prog{IMSL} was not
available for testing. The calls have been left in to facilitate later
work on full \prog{IMSL} support. In their current state, they are
definitely incomplete and probably not functional.} \prog{LAPACK}
routines are free and available from several sources, e.g.\ by
\prog{ftp} from \prog{netlib.att.com}\footnote{Login as
\prog{anonymous} and use your e-mail address as password.} or by e-mail
from \prog{netlib@ornl.gov}\footnote{Send a mail with the single word
``help'' to that address to get detailed information.}. The \prog{HYDROLIB}
archive file contains all the necessary files from \prog{LAPACK} and
\prog{BLAS} (a set of low-level routines used by \prog{LAPACK}).
However, you might wish to install a newer version or a version optimized
for your machine.

Linear algebra functions are needed for the calculation of eigenvalues
(which is done only during initialization and for small matrices, so
efficiency is not important) and for solving systems of linear
equations. For some types of calculation, a matrix inversion routine
is also called. Users who have experience with linear algebra packages
and with the C preprocessor should be able to add support for other
libraries. To do this, check all occurrences of
\prog{\#ifdef LAPACK} in the source files. Add an appropriate
section for whatever library functions you want to use, and delimit it
by \prog{\#ifdef} and \prog{\#endif} with a suitable label. Replace
all \prog{\#define LAPACK} lines in the file \prog{config.h} and in
your application programs with a definition of the new label you have
chosen. Add an appropriate section in the script file \prog{install\_lib}
by copying and modifying the one for \prog{LAPACK}. Then reinstall the
library.

Porting \prog{HYDROLIB} to other operating systems is certainly
possible, but some knowledge of Unix and the target system is
required. However, porting the library is not necessary for running
working programs on other machines, since it is possible to create
``clean'' Fortran-77 code on a Unix system that can then be used
wherever a Fortran compiler is available.


\subsection{Installation}

\begin{enumerate}

\item
Edit the file \prog{local.def}. The supplied version already contains
the correct configurations for the most common Unix systems. If you are
using one of these systems, simply uncomment the respective lines. For
other machines, you must find the correct values from the manual for
the Fortran compiler. The values of the variables have the following
meaning:
\begin{description}
\item[\prog{FC}] The name of the command that invokes the Fortran compiler.
\item[\prog{RANLIB}] The name of the command that creates a table of
contents for archives. If your system has a command called \prog{ranlib} (check
with \prog{which ranlib}), use it. Otherwise, try \prog{touch} (which actually
does nothing). If this leads to error messages during linking, check the
man page for the \prog{ar} command.
\item[\prog{FCFLAGS1}] The options for compilation of a single file without
linking. These options are used to compile the invariable parts of the
library during installation.
\item[\prog{FCFLAGS2}] The options for compilation and linking of one
or several Fortran files. These options are used when compiling
a program that uses \prog{HYDROLIB}.
\item[\prog{cpp}] This variable should have the value \prog{1} if
the Fortran compiler understands the option \prog{-cpp}, which
causes it to run programs through the C preprocessor before compiling
them, and if your Fortran compiler accepts the non-standard statement
``implicit none''. Otherwise it should have the value \prog{0}.
\item[\prog{dpoption}] If your Fortran compiler has an option to
convert single precision programs to double precision, this option
should become the value of this variable. Otherwise the value must
be \prog{NONE}.
\end{description}

At the end of \prog{local.def}, there is a line starting with
\prog{set lalib =}. You need to change this line only if you
use another linear algebra library than \prog{LAPACK} or another
version of \prog{LAPACK} than the one supplied with \prog{HYDROLIB}.
Change the contents of the following string to
whatever compiler options are necessary on your system to make the
compiler find the linear algebra libraries.

\item
Type\\
\wsp\prog{install\_lib}\\
This will initiate the installation procedure. Installation might take several
minutes, and in the course of it a series of Fortran compiler calls will
be displayed.

\item
To check the installation, type\\
\wsp\prog{fc example.f}\\
This will compile a short test program. After
compilation is finished, type\\
\wsp\prog{a.out}\\
to run the test program. The ouptut should be\\
\wsp\prog{N = 6}\\
\wsp\prog{3.288792419691321   4.296910545080898}\\
\wsp\prog{9.912899454371797   94.71505251108327}\\
The accuracy of the printed results will be system dependent.

\end{enumerate}

\subsection{Preparing application programs}

To use the library in an application program, insert the following two lines
before the first subroutine, function, or program statement:\\
\wsp\prog{\#include "}{\em program}\prog{.h"}\\\
\wsp\prog{\#include <subr.f>}\\
{\em program} should be the name of the program file without the
extension \prog{.f}. The file {\em program}\prog{.h} should be a
copy of \prog{config.h} in which the parameters have been modified
appropriately (see section~\ref{sec:config} for details).
Note that the character \prog{\#} in the lines above may not be
preceeded by white space.

The available library calls are explained in section~\ref{sec:imp}.

\subsection{Compilation and Fortran source code generation}

Two commands are supplied to handle compilation and generation of
portable Fortran source code.  The command \prog{fc} will compile an
application program using all necessary compiler options and link it
with the library. File arguments and additional compiler options
should be specified as for the standard \prog{f77} command. The
name of the configuration file can be specified using the option
\prog{-h}~{\em config-file}\/. If this option is missing, \prog{fc}
will use the name of the first source code file given and
replace its extension \prog{.f} by \prog{.h}. If no file with
this name exists, \prog{fc} will issue a warning and assume
the name \prog{config.h}.

The command \prog{fprog} will produce a complete Fortran source
program that contains all necessary code (except for routines from the
linear algebra package) and can be ported to any machine with a
Fortran 77 compiler. It accepts the name of a single file with
extension \prog{.f} and produces a file with the same name and
extension \prog{.fsrc}. To run this program on another machine,
transfer the generated program file {\em and} the files \prog{z2cl*},
\prog{ocoeff} and \prog{hcoeff} from the folder \prog{hydrolib}
to this machine. 

\section{Configuration files}
\label{sec:config}

The Hydrodynamic Interactions Library requires some numerical parameters
and some options that determine what is to be calculated. A special
configuration file contains all this information. An example of such
a file is the file \prog{config.h} supplied with the library. For each
application program, create a copy of this file and modify it according
to your needs.

\subsection{Numerical parameters}

These are specified by lines of the type\\
\wsp\prog{\#define }{\em parameter\_name value}

\begin{description}

\item[]
\prog{\_NP\_}\\
This parameter must be set to the number of particles.

\item[]
\prog{\_LM\_}\\
This parameter must be set to the value of $l_{\rm max}$, described in
section~\ref{sec:numcalc}. It must be a number between 0 and 3.

\end{description}


\subsection{Compile-time options}

Compile-time options are used to adapt the program code to the
requirements of the application and the programming environment. They
all result in different sections of code being included in the final
program. An option is activated by a line of the type\\
\wsp\prog{\#define }{\em option\_name}\\
and switched off by\\
\wsp\prog{\#undef }{\em option\_name}\\
(In principle lines starting with \prog{\#undef} could be completely
removed, but for ease of modification this is not recommended.)

The first set of compile-time options is related to the type of
problem being solved:

\begin{description}

\item[]
\prog{PERIODIC}\\
If defined, hydrodynamic interactions are calculated for a
system with periodic boundary conditions. Only cubic elementary cells
are supported. A common block named \prog{pbc} contains a variable
\prog{box} of type \prog{real} which must be set to the edge length of
the cubic cell. The coordinates that specify the particle
configuration must be positive and less than the edge length. The
particle configuration must be consistent with periodic boundary
conditions, i.e.\ the particles may not overlap with each other or
with each other's periodic images. This is {\em not} checked by the
program.\\
If \prog{PERIODIC} is not defined, a finite assembly of
spheres in an infinite fluid is assumed. The particles may not overlap
with each other; again this is {\em not} checked by the program. The
particle coordinates can otherwise be arbitrary.

\item[]
\prog{RIGID}\\
If defined, \prog{HYDROLIB} is able to handle
rigid connections between the particles. A ``rigid connection'' between
two particles means that no relative motion, neither translational nor
rotational, is possible between these particles. Other than that, the
``connections'' have no influence on the calculations; they are assumed
to have no masses and no influence on the fluid flow. From this definition it
follows that connectivity is transitive: if particle~1 is connected with
particle~2 and particle~3, there is also an implicit connection between
particles~2 and 3, since no relative motion between them is possible.
The specification of connections is explained in section~\ref{sec:imp}.\\
The presence of connections means that the number of degrees of freedom
of the system is reduced, since any cluster of interconnected particles
has only 6 degrees of freedom in total. Consequently, the friction and
mobility matrices have fewer elements, namely 6 times the number of
rigid clusters in the system. A particle cluster is described by the
translational velocity of its ``center of mass'' (or rather what would
be its center of mass if all particles had identical mass; in fact masses
never enter in any calculation) and by an angular velocity about this
point\footnote{In principle any point could be chosen as reference
point for describing clusters, but this choice makes the interpretation
of results easier if the cluster happens to have some symmetry.}.
To make use of the results, the application program needs some information
about the rigid clusters, which are available in a common block named
\prog{rb}, which has the following form:\\
\wsp\prog{real vrb(6*\_NP\_,6*\_NP\_)}\\
\wsp\prog{integer nrb,irb(\_NP\_)}\\
\wsp\prog{common /rb/ vrb,nrb,irb}\\
The variable \prog{nrb} contains the number of rigid clusters. The
array \prog{irb} contains for each particle the number of the cluster
it belongs to (i.e.\ a number between 1 and \prog{nrb}). All particles
with the same number form a cluster. The array \prog{vrb} makes it
possible to convert forces and velocities between the rigid-cluster
description (\prog{6*nrb} degrees of freedom) and the single-sphere
description (6$N$ variables). Its first index runs from 1 to $6N$, its
second index from 1 to \prog{6*nrb}. Multiplying a \prog{6*nrb}-dimensional
velocity vector from the right gives the individual particle velocities,
whereas multiplying by a 6$N$-dimensional force vector from the left gives
the total forces and torques acting on the rigid clusters.\\

\item[]
\prog{FIXED}\\
If defined, mobility calculations can be performed on systems
in which some spheres are fixed while others can freely move.
An additional common block named \prog{fix} is defined,
which has the following form:\\
\wsp\prog{integer nfixed}\\
\wsp\prog{logical full}\\
\wsp\prog{common /fix/ nfixed,full}\\
The variable \prog{nfixed} contains the number of fixed particles.
The fixed particles are always the {\em last} \prog{nfixed}
particles in the configuration. If the variable \prog{full} has
the value \prog{.true.}, everything will be recalculated.
If it has the value \prog{.false.}, the program will assume that
the fixed particles have not moved since the evaluation for the
previous configuration. The variable \prog{full} must always be
set to \prog{.true.} initially.

\item[]
\prog{FIXED\_FAST}\\
Identical to \prog{FIXED}, except that the calculations are
performed faster at the expense of a higher memory requirement.

\item[]
\prog{LUBRICATION}\\
If defined, the program adds short-range corrections calculated from
lubrication theory to the results of the multipole
expansion. This should always be done unless it is known that these
corrections are negligible.\\
A potential problem is caused by the fact that the short-range
corrections diverge for touching spheres. This will cause loss of
precision for short distances and a program crash for actually
touching spheres. The latter can be avoided by enabling the option
\prog{RIGID} and defining a rigid connection between the touching
particles (this does not affect the results since with stick
boundary conditions, there can never be any relative motion between
touching spheres). Specifying rigid connections also removes the
loss of precision at small non-zero distances, but in that case the
connection will also lead to different results.

\item[]
\prog{FRICTION}\\
If defined, the program calculates the friction matrix. The friction
matrix is returned in a common block named \prog{frict} of the form\\
\wsp\prog{real fr(6*\_NP\_,6*\_NP\_)}\\
\wsp\prog{common /frict/ fr}\\
\begin{sloppy}
The friction matrix consists of $6\times 6$-submatrices for each particle
pair. It is always symmetric. The order of the 6 coordinates is
$(x,y,z)_{\rm Translation},$ $(x,y,z)_{\rm Rotation}$.\\
\end{sloppy}
If the option \prog{RIGID} is used (see description above), the friction
matrix has dimensions \prog{6*nrb}$\times$\prog{6*nrb}. Since
\prog{nrb} is calculated at runtime, the array containing the friction
matrix is still dimensioned as $6N\times 6N$, but not all elements
are used.

\item[]
\prog{MOBILITY}\\
If defined, the program calculates the mobility matrix. The mobility
matrix is returned in a common block named \prog{mobil} of the form\\
\wsp\prog{real mo(6*\_NP\_,6*\_NP\_)}\\
\wsp\prog{common /mobil/ mo}\\
\begin{sloppy}
The mobility matrix consists of $6\times 6$-submatrices for each particle
pair. It is always symmetric. The order of the 6 coordinates is
$(x,y,z)_{\rm Translation},$ $(x,y,z)_{\rm Rotation}$.\\
\end{sloppy}
If the option \prog{RIGID} is used (see description above), the mobility
matrix has dimensions \prog{6*nrb}$\times$\prog{6*nrb}. Since
\prog{nrb} is calculated at runtime, the array containing the mobility
matrix is still dimensioned as $6N\times 6N$, but not all elements
are used.\\
If the option \prog{FIXED} is used, only the mobility matrix elements
for the freely moving particles have meaningful values.

\item[]
\prog{VELOCITIES}\\
If defined, the program calculates particle velocities for a given set of
forces and torques. The forces (input) and velocities (output) are
contained in a common block named \prog{fv} of the form\\
\wsp\prog{real f(6*\_NP\_),v(6*\_NP\_)}\\
\wsp\prog{common /fv/ f,v}\\
\begin{sloppy}
The arrangement of the components of \prog{f} is
$F_x^{(1)},F_y^{(1)},F_z^{(1)},$ $T_x^{(1)},T_y^{(1)},T_z^{(1)},
\ldots,$ $T_z^{(N)}$. Similarly, \prog{v} is arranged as
$U_x^{(1)},U_y^{(1)},U_z^{(1)},$ $\Omega_x^{(1)},\Omega_y^{(1)},
\Omega_z^{(1)},\ldots,$ $\Omega_z^{(N)}$.\\
\end{sloppy}
If the option \prog{RIGID} is used, the forces and velocities are still
single-particle quantities, but of course their calculation procedes
in a different way. If the rigid-body velocities of the particle
clusters are needed, the option \prog{RB\_VELOCITIES} (explained below)
should be used.\\
If the option \prog{FIXED} is used, only the velocity components
referring to the freely moving particles have meaningful values.

\item[]
\prog{RB\_VELOCITIES}\\
This option must be combined with \prog{RIGID} (see above).
It is similar to
\prog{VELOCITIES}, except that input forces and output velocities 
describe the rigid-body motion of particle clusters. Consequently
only the first \prog{6*nrb} elements are used.

\item[]
\prog{NORMALIZE}\\
If defined, the friction and mobility matrices are normalized by their
one-particle values. Since for a single particle there is no coupling
between translation and rotation, normalization is possible only for
the pure translation and rotation submatrices $\tzeta^{tt}$,
$\tzeta^{rr}$, $\tmu^{tt}$ and $\tmu^{rr}$. Consequently it should
{\em not} be done if combinations of translational and rotational
motion are to be used.\\
Particle velocities calculated using the option \prog{VELOCITIES} are
never normalized.

\item[]
\prog{POSDEF}\\
The system of linear equations that is solved to obtain the friction
or mobility matrix can be shown to be positive definite. However, for
large systems the coefficient matrix may not be positive definite from
a numerical point of view. Linear algebra packages might therefore
issue error messages when special algorithms for positive definite
matrices are used.  Therefore the decision whether or not to use these
algorithms is left to the user.\\ If \prog{POSDEF} is defined,
routines from \prog{LAPACK} are called that are optimized for positive
definite matrices. Otherwise, routines for general symmetric matrices
are used.  It is recommended to leave \prog{POSDEF} defined until an
actual warning about indefinite matrices occurs.\\

\end{description}

The second set of compile-time options is related to the programming
environment:

\begin{description}

\item[]
\prog{PACKED}\\
If defined, the coefficient matrix for the system of linear equations
to be solved is stored in a packed format suitable for symmetric
matrices. This saves almost 50\% of memory space, but can lead to
slightly higher execution times. In most applications the memory
saving will outweigh the increased execution time, therefore it is
recommended to leave this option enabled. In some cases, a
small {\em decrease} of execution time was even observed with packed
storage.

\item[]
\prog{DP}\\
If defined, double precision code is used everywhere, and all external
library functions are called in their version for double-precision
calculations. If your Fortran compiler has an option for treating
single-precision code as double precision (most compilers have), this
will be applied to your application program.  In this case, defining
and undefining \prog{DP} is sufficient to switch between single and
double precision versions. If your compiler does not have such an
option, you must prepare your code manually for double-precision
calculations. However, there is no need to ever modify \prog{HYDROLIB}
code.

\item[]
\prog{LAPACK}\\
If defined, routines from the linear algebra package \prog{LAPACK} are
used for the calculation of eigenvalues, for the solution of a system
of linear equations, and for matrix inversion.\\
Note that currently no other linear algebra libraries are
supported. If you add support for another library, you can
define the label you have chosen for that library instead
of \prog{LAPACK}.

\end{description}


\section{Fortran subroutines in the library}
\label{sec:imp}

To calculate hydrodynamic interactions for a certain configuration of
particles, you must write a Fortran program that calls subroutines
from \prog{HYDROLIB}. For illustrational
purposes, an example program (\prog{example.f}) is included that
calculates the four independent friction coefficients for a linear
chain of touching particles.

During the initialization of the program, the subroutine \prog{init}
must be called. If periodic boundary conditions are used, the
variable \prog{box} in the common block \prog{pbc} must be set to the
edge length of the cubic box, as explained in section~\ref{sec:config}.

After that, the subroutine \prog{eval} can be called any number of times.
This subroutine has no arguments, all input and output data are
passed via common blocks. Which quantities are calculated by
\prog{eval} depends on the compile-time options used, see
section~\ref{sec:config} for details.

Before calling \prog{eval}, the particle configuration must be
stored in a two-dimensional array in the
common block \prog{conf}. The first dimension of this array is 3, for
the $x$-, $y$-, and $z$-coordinates, the second dimension is the
number of particles. An example of a definition for this common block
is\\
\wsp\prog{real c(3,\_NP\_)}\\
\wsp\prog{common /conf/ c}\\
If the option \prog{RIGID} is activated, the particle configuration
must be extended by a list of rigid connections between the particles.
This takes the form of a $N\times N$-array of \prog{character*1}. Any
character other than a space stands for a rigid connection between
two particles. Only the upper-diagonal part of this array, i.e.\ elements
for which the second index is larger than the first one, are used.
With this addition, the common block \prog{conf} takes the form\\
\wsp\prog{real c(3,\_NP\_)}\\
\wsp\prog{character*1 cnct(\_NP\_,\_NP\_)}\\
\wsp\prog{common /conf/ c,cnct}\\

When particle velocities are to be calculated (option
\prog{VELOCITIES}), the forces must be defined before calling
\prog{eval}. This is done in the common block
\prog{fv} as explained in section~\ref{sec:config}.

After calling \prog{eval}, the results can be found in various common
blocks as explained in section~\ref{sec:config}. The input data
are not changed, with the possible exception of the character array
specifying rigid connections. The contents of this array may have been
changed to another equivalent specification of the same set of
connections.


\begin{thebibliography}{9}

\bibitem{cifehiwabl}
  \name{B. Cichocki, B.U. Felderhof, K. Hinsen,
        E. Wajnryb, J. B{\l}awzdziewicz},
  Friction and mobility of many spheres in Stokes flow,
  J. Chem. Phys. \vol{100}, 3780~(1994)

\bibitem{hi1}
  \name{K. Hinsen},
  \prog{HYDROLIB}: a library for the evaluation of hydrodynamic
       interactions in colloidal suspensions,
  Computer Physics Communications (1995)

\end{thebibliography}

\end{document}
